%导言区
\documentclass[UTF8]{ctexart}	%ctexbook,ctexreport,没有letter类型 

\usepackage{graphics}
\usepackage{indentfirst}
\usepackage{enumerate}
\usepackage{listings}
\usepackage{xcolor}
\lstset{
    language=C++,
    basicstyle=\linespread{0.8}\ttfamily,
    tabsize=4,
    numbers=left,
    numberstyle=\tiny\color{gray},
    breaklines=true,
    frame=single,
}
\setlength{\parindent}{2em}
\newcommand\degree{^\circ} %定义degree

\title{\heiti 一种新的数据结构----TT树} %指定字体:黑体
\author{\kaishu 顾顺}  %指定字体:楷书
\date{\today}


% 正文区 (文稿区)
\begin{document}
	\maketitle    %显示出标题内容
	\tableofcontents  %生成目录
	\begin{abstract}
		\textbf{关键词: }数据结构\space 平衡树 \space 二叉索引树 \space 自平衡
	\end{abstract}
	
	\newpage

	\section{引言}
	\space\space 平衡二叉索引树是一种用途广泛的数据结构,可以实现快速插入,删除,查询排名,第k大等,同时也是C++语言中set与map的底层实现(红黑树).
	\\\\
	\space\space 本文将提出一种自平衡二叉索引树,称之为Twist Tree(TT).
	\\\\
	\space\space 该结构没有采用一般的平衡方式旋转(rotation),而是用了一种全新的维护方式Twist.
	
	\newpage

	\section{朴素的TT结构}
	\subsection{核心算法介绍}
		这里仅介绍插入和删除算法,其他同普通二叉索引树
		\\
		\indent 首先定义一棵合法的TT树,当且仅当满足如下条件:
			\begin{enumerate}[\indent(1)\space]

			\item 满足二叉索引树的所有条件

			\item 左右子树的大小(size)相差不超过1

			\item 左右子树均为空结点或一颗合法TT树

			\end{enumerate}

		维护方式比较简单.以插入为例,假设需要向一棵合法TT树的左子树插入一个键值(未出现过),那么可以分为以下两种情况:
			\begin{enumerate}[\indent(1)\space]

			\item 插入后,左右子树大小相差依然不超过1.这种情况下,只需要向左插入并保证左子树依然是合法TT树,显然可以直接递归

			\item 若插入后破坏了条件2,那么显然,左子树大小比右子树大2,那么可以进行如下操作:向右子树插入当前根节点的键值,将当前根节点键值替换为左子树最大值,删去左子树最小值.这些操作可以直接递归,这就是朴素的Twist(缠绕)操作.

			\end{enumerate}
		删除同理,不再赘述
	\subsection{时间复杂度分析}
		以下是对于插入及删除的时间复杂度分析.由于实现原理上类似,假定插入和删除的复杂度同级.事实上,可以通过计算得知确实如此.
		\\
		\indent 设$i(p,v),d(p,v)$表示在p为根的子树上插入/删除v,那么在递归的过程中,显然有:\\ 
		\indent 左子树大小为奇数,偶数等概率,右子树同理,于是可知上文中,情况(1)出现概率为$\frac34$,情况(2)出现概率为$\frac14$.
		\\ \indent 记:T(n)表示i(p,v)\space (p子树大小为n)的复杂度,那么有递推式:
		$$T(n)=\frac14\cdot\space3T(\frac n2)+\frac34\cdot\space T(\frac n2)+O(1)$$
		根据主定理:
		$$T(n)=\Theta(n^{log_2{\frac32}})\approx\Theta(n^{0.5849625})$$
	\subsection{代码实现(C++,插入函数)}
		\begin{lstlisting}
const int MAXN=1e5+5;
const int INF=INT_MAX;
class TT_node{
	friend class simple_TT;
	private:
		int left,right,v,sz,fr,to;
		TT_node(void){
			left=right=0;
			sz=0;
			fr=to=INF;
		}
		void operator=(TT_node x){
			v=x.v;
			fr=x.fr;
			to=x.to;
		}
};
class simple_TT{
	private:
		int n;
		int root;
		TT_node tr[MAXN];
		int new_node(int v){
			n++;
			tr[n]={0,0,v,1,pre(v),suc(v)};
			return n;
		}
		int new_node(void){
			n++;return n;
		}
		void pushup(int p){
			tr[p].sz=tr[tr[p].left].sz+tr[tr[p].right].sz+1;
		}
	public:
		const int pre(const int v){
			//......
			int p;
			return p;
		}
		const int suc(const int v){
			//.......
			int p;
			return p;
		}
		void ins(int& p=root,const int id,const int v){
			if(p==0){
				p=new_node();
				tr[p]=tr[id];
				if(tr[p].fr) tr[tr[p].fr].to=n;
				if(tr[p].to) tr[tr[p].to].fr=n;
				return;
			}
			if(v<tr[p].v){
				if(tr[tr[p].left].sz>tr[tr[p].right].sz){
					ins(tr[p].right,p,tr[p].v);
					if(tr[p].fr) tr[tr[p].fr].to=n;
					if(tr[p].to) tr[tr[p].to].fr=n;
					int f=tr[p].fr;
					tr[p]=tr[f];
					del(tr[p].left,f,tr[f].v);
					if(tr[p].fr) tr[tr[p].fr].to=p;
					if(tr[p].to) tr[tr[p].to].fr=p;
				}
				ins(tr[p].left,id,v);
			}else{
				if(tr[tr[p].left].sz<tr[tr[p].right].sz){
					ins(tr[p].left,p,tr[p].v);
					if(tr[p].fr) tr[tr[p].fr].to=n;
					if(tr[p].to) tr[tr[p].to].fr=n;
					int f=tr[p].to;
					tr[p]=tr[f];
					del(tr[p].right,f,tr[f].v);
					if(tr[p].fr) tr[tr[p].fr].to=p;
					if(tr[p].to) tr[tr[p].to].fr=p;
				}
				ins(tr[p].left,id,v);
			}
			pushup(p);
		}
		
};

		\end{lstlisting}
	
	\newpage
	
	\section{TT插入删除的优化}
	\subsection{优化算法}
		\indent 在朴素算法的基础上,增加函数$i\_d(p,vi,vd)$,表示同时在以p为根的子树上删去$vd$并插入$vi$,达到降低时间复杂度的目的.
		\\
		\indent 实现方面,$i\_d$函数相当于插入和删除函数的复合,只需要将同一棵子树上的插入和删除合并即可.
	\subsection{时间复杂度分析}
		\indent 类似的,假设插入和删除函数复杂度同级,为$T1(n)$,$i\_d$函数复杂度为$T2(n)$,通过类似于朴素算法的分析,可以得到递推式:
		$$
		T1(n)=T1(\frac n2)+\frac14\space T2(\frac n2)+O(1)
		$$$$
		T2(n)=\frac34\space T1(\frac n2)+\frac34\space T2(\frac n2)+O(1)
		$$


		记$a_n=T1(2^n),b_n=T2(2^n)$ 可以设计出:
		$$c_n=a_n+\frac{-1+\sqrt{13}}6b_n
		\\
		d_n=a_n+\frac{-1-\sqrt{13}}6b_n
		$$

		于是有
		$$c_n\approx(\frac{7+\sqrt{13}}8)^n
		$$$$
		d_n\approx(\frac{7-\sqrt{13}}8)^n
		$$
		
		得到
		$$
		T1(n)=\Theta(n^{log_2\frac{7+\sqrt{13}}8})\approx \Theta(n^{0.4067477})
		$$

	\newpage

	\section{TT的特殊性质及一些应用}
	\subsection{性质}
		\indent 首先是结构上的性质.
		\\
		\indent 显然,任何一对左右子树大小相差不超过1,那么可知树高严格$\lceil logn\rceil$,也是完美平衡的(任何叶子高度相差不超过1).
		\\ \\
		\indent 更重要的是Twist操作的特殊性.与旋转不同,Twist操作每次只会改变不超过3个结点,而无关节点的绝对位置是没有变化的,也就意味着\space TT树支持堆式存储(p的左儿子为$p*2$,右儿子为$p*2+1$).堆式存储具有实现简单,空间小,速度快等优点,一定程度上弥补了其插入删除复杂度稍高(约$n^{0.4}$)的缺点.
		\\
		\indent 同时,TT树也是笔者知道的唯一支持堆式存储的二叉索引树.
	
	\newpage
	
	\section{结语}
		\indent TT树作为二叉索引树,插入删除复杂度稍高,但其严格树高,特别是支持堆式存储的优点,使它可以用于静态存储,而无需指针结构.
	
	% 引用论文
	\begin{thebibliography}{99}  
		无
	
	\end{thebibliography}
	

\end{document}



